\documentclass[ngerman]{dtk}
\ifluatex\else
\usepackage[utf8]{inputenc}
%\usepackage[latin9]{inputenc}
\fi
\usepackage{siunitx}
\usepackage{framed}
\usepackage{booktabs}
%\usepackage{etoolbox}
%\usepackage{keyval}
\usepackage{cleveref}
%----------------------
\newenvironment{bsp}{\begin{framed}\begin{footnotesize}}%
{\end{footnotesize}\end{framed}}
%----------------------
\newcommand\catalogueentry[1]{%
\RaggedRight\begingroup
\setkeys{catalogue}{#1}%
\ifdef{\KVhouse}{\section{\KVhouse
			\ifdef{\KVlabel}{\label{\KVlabel}}{}}
			}{}%
\begin{labeling}{Beschreibung}
%	\footnotesize %
%	\setlength{\itemsep}{0pt}%
%    \setlength{\parskip}{0pt}%
%    \setlength{\parsep}{0pt}%
\ifdef{\KVdescription}{\item[Beschreibung] \KVdescription}{}%
\ifdef{\KVlocation}{\item[Verortung] \KVlocation}{}%
\ifdef{\KVinterior}{%
	\item[Ausstattung] \KVinterior %
	\ifboolexpr{bool{@KVinteriorM} or bool {@KVinteriorW} or bool {@KVinteriorS}}{%
	\begin{labeling}{Wandgemälde}
%		\setlength{\itemsep}{0pt}%
%    		\setlength{\parskip}{0pt}%
%    		\setlength{\parsep}{0pt}%
			\ifdef{\KVinteriorM}{\item[Mosaike:] \KVinteriorM}{}
			\ifdef{\KVinteriorW}{\item[Wandgemälde:] \KVinteriorW}{}
			\ifdef{\KVinteriorS}{\item[Statuen:] \KVinteriorS}{}
			\end{labeling}
		}{}}%
	{%
	\ifboolexpr{bool{@KVinteriorM} or bool {@KVinteriorW} or bool {@KVinteriorS}}{%
		\item[Ausstattung]%
	\begin{labeling}{Wandgemälde}	
%			\setlength{\itemsep}{0pt}
%	 		\setlength{\parskip}{0pt}%
%    			\setlength{\parsep}{0pt}%
				\ifdef{\KVinteriorM}{\item[Mosaike:] \KVinteriorM}{}
				\ifdef{\KVinteriorW}{\item[Wandgemälde:] \KVinteriorW}{}
				\ifdef{\KVinteriorS}{\item[Statuen:] \KVinteriorS}{}
			\end{labeling}
	}{}}%
\ifdef{\KVsize}{	\item[Größe] \SI{\KVsize}{\meter\squared}}{}%
\ifdef{\KVliterature}{\item[Erwähnungen]S.\ \KVliterature}{}%		
\end{labeling}
\endgroup
}

\makeatletter
\newbool{@KVinteriorM}%Mosaik
\newbool{@KVinteriorW}%Wandgemälde
\newbool{@KVinteriorS}%Statue
\define@key{catalogue}{house}{\def\KVhouse{#1}}
\define@key{catalogue}{label}{\def\KVlabel{#1}}
\define@key{catalogue}{description}{\def\KVdescription{#1}}
\define@key{catalogue}{location}{\def\KVlocation{#1}}
\define@key{catalogue}{size}{\def\KVsize{#1}}
\define@key{catalogue}{interior}{\def\KVinterior{#1}}%Innenausstattung
\define@key{catalogue}{interiorM}{\def\KVinteriorM{#1}\booltrue{@KVinteriorM}}
\define@key{catalogue}{interiorW}{\def\KVinteriorW{#1}\booltrue{@KVinteriorW}}
\define@key{catalogue}{interiorS}{\def\KVinteriorS{#1}\booltrue{@KVinteriorS}}
\define@key{catalogue}{literature}{\def\KVliterature{\ausgabe{#1}}}
%\define@key{catalogue}{literature}{%
%	\def\KVliterature{%
%				\ifdef{#1}%
%						{\ausgabe{#1}}
%						{\ifdef{\KVlabel}%
%							{\ausgabe{\KVlabel}}
%							{}%
%						}
%			}
%}

\makeatother
%----------------------python
\makeatletter
	\newwrite\myfile
	\immediate\openout\myfile=\jobname.sti
	\newcommand{\eintrag}[1]{\immediate\write\myfile{#1:\thepage}}
%	\newcommand{\ausgabe}[2]{#1 was referenced on page(s): #2}
	\newcommand{\ausgabe}[2]{#2}
	\newcommand{\stichworttabelle}{\IfFileExists{\jobname.tab}{\input{\jobname.tab}}{Tab file not found}}
\makeatother

\begin{document}
\title{Integration von Python in \XeLaTeX\ am Beispiel von Katalogeinträgen}
\Author{Lukas C.}{Bossert}%
			{Cranachstr.~24\\
			12157 Berlin\\
			\Email{lukas@digitales-altertum.de}}
\Author{Uwe}{Ziegenhagen}{Lokomotivstr. 9 \\ 50733~Köln}
\maketitle



Viele Dissertationen in der Archäologie enthalten am Ende der Arbeit einen Katalog, 
in dem die untersuchten Daten in einem bestimmten System aufgeschlüsselt präsentiert werden.
Ein solcher Katalog kann aus  Bildern, Bohrproben, Architekturelementen etc. bestehen.

Eine händische Erstellung der einzelnen Einträge bspw. über \Macro{section} oder \Macro{subsection} 
und anschließend in einer \Environment{items}-Umgebung ist nicht effizent, fehleranfällig und nur bei wenigen Katalogeinträgen einsetzbar.
Es muss zudem berücksichtigt werden, dass  dass die vorgegebenen Kategorien nicht für jeden Katalogeintrag passend sind, 
sodass  Kategorien leer bleiben  und dann im Katalogeintrag nicht auftauchen sollen.
Dabei soll der Code des Katalogeintrags möglichst viel redundante Tipp-Arbeit abnehmen, wie sie bspw. bei Maßeinheiten vorkommt.
Darüber hinaus sollen alle Einträge immer gleich formatiert sein und ihr Aussehen global verändert werden können.

Nachdem  eine  Lösung gefunden wurde, die allen bisher genannten Anforderungen entspricht (siehe unten),
sollten in einer Kategorie alle Seitenzahlen enthalten sein, auf denen im Haupttext auf den Katalogeintrag verwiesen wird.
Ein  Lösungsansatz sah die Nutzung von  \Package{glossaries} vor: 
Dafür musste allerdings für jeden Katalogeintrag ein eigenes Glossar angelegt werden, was nicht nur viel händische Arbeit bedeutet,
sondern auch die Zählerkapazitäten von \TeX\ überforderten.

Es musste also eine andere Lösung her und ich (Lukas) erinnerte mich an einen Beitrag, den Uwe Ziegenhagen in dtk 1/2015 vorgestellt hatte:
Damals ging es darum, bei einem Werkkatalog die Erwähnung des Stückes im Haupttext anzugeben. 
Prima, genau das, was ich gesucht habe und ich fand den damals verwendeten Code auf seiner Webseite.\footnote{\url{http://uweziegenhagen.de/?p=3020}}
Allerdings  sah ich nun, dass die Aufgabe  damals darin bestand, nur \emph{eine} Erwähnung im  Haupttext anzugeben,
was mit \Command{label} und \Command{ref{}} bewerkstelligt werden konnte.
Ich brauche jedoch alle Erwähnungen im Haupttext.
Meine Emailanfrage bei Uwe nach einer Möglichkeit mehr als nur eine Erwähnung anzeigen zu lassen,
beantwortete er mit dem Vorschlag, mittels Python das Problem zu lösen.

Um gemeinsam an dem hybriden Konstrukt von \XeLaTeX\ und Python arbeiten zu können, wurde ein Repository auf der 
Plattform \Program{github}\footnote{\url{https://github.com/LukasCBossert/DTK-TeX-Python}} erstellt, 
die unabhängig der Programmiersprache einen exzellenten Austausch und eine detaillierte Versionskontrolle bereithält,
sodass sich kollaboriertes Arbeiten  sehr gut konzipieren lässt. 


\subsection{Katalogeintrag}
Als Aufgabe soll ein Katalog zu den Häusern in Pompeji angelegt werden,
der Auskunft über den Namen des Hauses, dessen Verortung und Grundstücksgröße  geben soll.
Zudem soll eine kurze Beschreibung enthalten sein und die Angabe über die Innenausstattung, 
die wiederum aufgeschlüsselt werden kann auf die Untergruppen Mosaik, Wandgemälde und Skulptur. 
Zum Schluss soll im Katalogeintrag angezeigt werden, 
auf welchen Seiten des Fließtextes das Haus genannt wird.

Die gefundene Lösung für die Umsetzung der Anforderungen funktioniert mit dem Paket \Package{keyval}.\footnote{Basierend auf der Idee vorgestellt auf: \url{http://tex.stackexchange.com/a/254336/98739}}
Dafür werden in der Präambel verschiedene Schlüssel (\Command{keys}) definiert.
In der Grundversion sieht die Definition eines Eintrages wie folgt aus:\footnote{Vgl. \url{http://www.tug.org/tugboat/tb30-1/tb94wright-keyval.pdf}}
\begin{lstlisting}[style=noNumber]
\define@key{family}{key}{#1}
\end{lstlisting}
Für das konkrete Beispiel wird die \Command{key}-Familie (\Command{family}) mit \Command{catalogue} angegeben,
der Schlüssel (\Command{key}), was einer Kategorie im Katalog entspricht, 
als \Command{house} bezeichnet und als Resultat soll der Wert im Makro \Macro{KVhouse} gespeichert werden:
\begin{lstlisting}[style=noNumber]
\define@key{catalogue}{house}{\def\KVhouse{#1}}
\end{lstlisting}
Dem Beispiel entsprechend können alle Kategorien als Schlüssel angelegt werden:
\begin{lstlisting}[style=number]
\makeatletter
\define@key{catalogue}{house}{\def\KVhouse{#1}}
\define@key{catalogue}{label}{\def\KVlabel{#1}}
\define@key{catalogue}{description}{\def\KVdescription{#1}}
\define@key{catalogue}{location}{\def\KVlocation{#1}}
\define@key{catalogue}{size}{\def\KVsize{#1}}
\define@key{catalogue}{interior}{\def\KVinterior{#1}}
\define@key{catalogue}{interiorM}{\def\KVinteriorM{#1}}
\define@key{catalogue}{interiorW}{\def\KVinteriorW{#1}}
\define@key{catalogue}{interiorS}{\def\KVinteriorS{#1}}
\makeatother
\end{lstlisting}
Das Aussehen eines Katalogeintrages wird separat mit dem Makro
 \Macro{newcommand}\Macro{catalogueentry}[1]{\ldots} 
definiert:
Darin wird zunächst festgelegt,
dass eine neue Gruppe beginnt (\Macro{begingroup}), sodass
es zu keinen Problemen mit den jeweils definierten Schlüsseln kommt, 
da diese für jeden Katalogeintrag neu definiert werden.
Dann sollen die  Einträge im Flattersatz gesetzt werden (\Macro{RaggedRight}) und schließlich die Angabe,
 welche Schlüsselfamilie (hier \Command{catalogue}) auszulesen ist.
\begin{lstlisting}[style=number]
\newcommand\catalogueentry[1]{%
\begingroup
\RaggedRight
\setkeys{catalogue}{#1}
...
\end{lstlisting}

Es folgt in der Definition die Verarbeitung der einzelnen Schlüssel:
Da nur die Ausgabe einer Kategorie erfolgen soll,
wenn diese auch mit Informationen versehen ist,
wird dies über die Abfrage \Macro{ifdef} erledigt.
Der Inhalt der Kategorie \Command{house} wird als \Macro{section}-Titel verwendet und, wenn vorhanden, mit einem \Macro{label} versehen.
Die weiteren Kategorien sollen in einer \Environment{labeling}-Umgebung aufgelistet werden.
Die Definition wird mit \Macro{endgroup} geschlossen.
\begin{lstlisting}[style=number]
...
\ifdef{\KVhouse}{\section{\KVhouse
			\ifdef{\KVlabel}{\label{\KVlabel}}{}}}{}
\begin{labeling}{Beschreibung}
	\ifdef{\KVdescription}{\item[Beschreibung] \KVdescription}{}
	\ifdef{\KVlocation}{\item[Verortung] \KVlocation}{}
	\ifdef{\KVinterior}{\item[Ausstattung] \KVinterior
			\begin{labeling}{Wandgemälde}
				\ifdef{\KVinteriorM}{\item[Mosaike:] \KVinteriorM}{}
				\ifdef{\KVinteriorW}{\item[Wandgemälde:] \KVinteriorW}{}
				\ifdef{\KVinteriorS}{\item[Statuen:] \KVinteriorS}{}
			\end{labeling}
			}{}
	\ifdef{\KVsize}{\item[Größe] \SI{\KVsize}{\meter\squared}}{}
\end{labeling}
\endgroup
}
\end{lstlisting}

Mit dieser Einstellung lassen sich die Katalogeinträge schon sehr gut darstellen.
Im Fließtext des Hauptdokuments kann man an gewünschter Stelle mit \Macro{catalogueentry} einen Katalogeintrag eintragen.
So wird aus den folgenden Angaben 

\begin{lstlisting}[style=noNumber]
\catalogueentry{%
	house={Haus des M. Fabius Rufus},
	label={haus:M-Fabius-Rufus},
	size={172},
	description={Haus besteht aus mehreren Einzelgebäuden.},
	location={Regio VII, Insula 16, Eingang 17--22.},
	interior={Reicher Fundkomplex.},
	interiorM={S/W-Mosaik},
	interiorW={Dionysius mit einer Mänade, Narzissus und ein Cupido, Hercules und Deinira, etc.},
	interiorS={Bronzene Statue eines Epheben},
}
\end{lstlisting}
ein Katalogeintrag, der so aussieht:\eintrag{Haus-Rufus}
\begin{bsp}
\catalogueentry{%
	house={Haus des M. Fabius Rufus},
	label={haus:M-Fabius-Rufus},
	size={172},
	description={Haus besteht aus mehreren Einzelgebäuden.},
	location={Regio VII, Insula 16, Eingang 17--22.},
	interior={Reicher Fundkomplex.},
	interiorM={S/W-Mosaik},
	interiorW={Dionysius mit einer Mänade, Narzissus und ein Cupido, Hercules und Deinira, etc.},
	interiorS={Bronzene Statue eines Epheben},
}
\end{bsp}
Wie man bei diesem Beispiel sieht, ist die Reihenfolge, in der die Kategorien
 angegeben werden, irrelevant, da die Definition in der Präambel entscheidend ist.
Der Wert bei \Command{size} wird intern sogleich an das vordefinierte \Macro{SI}-Makro mit entsprechender Einheit (\si{\meter\squared}) übergeben.
Ähnlich kann auch mit anderen Angaben verfahren werden (bspw. bei Abbildungen können die \Macro{label} an ein vordefiniertes Macro \Macro{cref} übergeben werden).

Bei dem oben beschriebenen Katalogeintrag zum 
›Haus des M.~Fabius Rufus‹ sind alle Kategorien ausgefüllt. \eintrag{Haus-Rufus}
Wenn eine Kategorie nicht ausgefüllt ist, wird sie nicht ausgegeben.
Allerdings besteht die Kategorie \Command{interior} [Ausstattung]  zusätzlich aus drei Unterkategorien 
(\Command{interiorM} [Mosaike], \Command{interiorW} [Wandgemälde], \Command{interiorS} [Statuen]).
Diese sollen auch dann angezeigt werden, wenn \Command{interior} selbst nicht definiert ist.

Ein solcher Fall tritt beim ›Haus des Wilden Ebers‹ auf. \eintrag{Haus-des-Wilden-Ebers}
Im Fließtext ist der Katalogeintrag wie folgt ausgefüllt:
\begin{lstlisting}[style=noNumber]
\catalogueentry{%
	house={Haus des Wilden Ebers},
	label={Haus-des-Wilden-Ebers2},
	size={54},
	description={Renovierung nach Erdbeben 62\,n.\,Chr.},
	location={Regio VII, Insula 4, Eingang 48, 43},
	interiorM={S/W-Mosaik},
	interiorW={Venus, Leda und der Schwan, Ariadne und Theseus},
}
\end{lstlisting}
So wird daraus:\eintrag{Haus-des-Wilden-Ebers}
\begin{bsp}
\catalogueentry{%
	house={Haus des Wilden Ebers},
	label={Haus-des-Wilden-Ebers},
	size={54},
	description={Renovierung nach Erdbeben 62\,n.\,Chr.},
	location={Regio VII, Insula 4, Eingang 48, 43},
	interiorM={S/W-Mosaik},
	interiorW={Venus, Leda und der Schwan, Ariadne und Theseus},
}
\end{bsp}
Um zu diesem Ergebnis zu kommen, 
ist es notwendig, mit Booleschen Operatoren zu arbeiten.
Dafür müssen bei den \Command{key}-Definitionen drei Operatoren eingeführt werden:
\begin{lstlisting}[style=number]
\newbool{@KVinteriorM}%Mosaik
\newbool{@KVinteriorW}%Wandgemälde
\newbool{@KVinteriorS}%Statue
\end{lstlisting}
Diese Booleschen Operatoren werden bei der Definition der einzelnen Einträge eingebaut und als 
\Command{true} gesetzt, wenn dieser Eintrag mit Informationen versehen wird.
Konkret sieht die Modifikation so aus:
 \begin{lstlisting}[style=number]
\define@key{catalogue}{interiorM}{\def\KVinteriorM{#1}\booltrue{@KVinteriorM}}
\define@key{catalogue}{interiorW}{\def\KVinteriorW{#1}\booltrue{@KVinteriorW}}
\define@key{catalogue}{interiorS}{\def\KVinteriorS{#1}\booltrue{@KVinteriorS}}
\end{lstlisting}

Zudem müssen die Booleschen Operatoren bei der Ausgabe der Katalogeinträge eingesetzt werden,
sodass geprüft wird (\Macro{ifboolexpr}), ob einer oder mehrere der Operatoren auf \Command{true} gesetzt ist.
\begin{lstlisting}[style=number]
\ifdef{\KVinterior}{%
	\item[Ausstattung] \KVinterior 
	\ifboolexpr{bool{@KVinteriorM} 
					or bool{@KVinteriorW} 
					or bool{@KVinteriorS}}{%
			\begin{labeling}{Wandgemälde}
				\ifdef{\KVinteriorM}{\item[Mosaike] \KVinteriorM}{}
				\ifdef{\KVinteriorW}{\item[Wandgemälde] \KVinteriorW}{}
				\ifdef{\KVinteriorS}{\item[Statuen] \KVinteriorS}{}
			\end{labeling}
			}{}}
	{\ifboolexpr{bool{@KVinteriorM} 
					or bool{@KVinteriorW} 
					or bool{@KVinteriorS}}{%
		\item[Ausstattung]%
	\begin{labeling}{Wandgemälde}	
				\ifdef{\KVinteriorM}{\item[Mosaike] \KVinteriorM}{}
				\ifdef{\KVinteriorW}{\item[Wandgemälde] \KVinteriorW}{}
				\ifdef{\KVinteriorS}{\item[Statuen] \KVinteriorS}{}
			\end{labeling}
	}{}}
\end{lstlisting}
Mit dieser modifizierten Ergänzung der Katalogeintragsdefinition werden die Unterkategorien von \Command{interior} ausgegeben, 
auch wenn \Command{interior} selbst nicht definiert ist.


Was noch fehlt ist die Ausgabe der Seiten, auf denen der Katalogeintrag
 erwähnt wird.
Dafür wird der Umweg über, oder besser gesagt die Integration von Python gewählt.
\section{Exkurs Python}
In der Präambel des \TeX-Dokuments muss aufgeführt werden, 
dass eine neue Datei geschrieben wird, die den gleichen Namen hat, aber die Endung \Command{.sti}.
In diese Datei werden alle Stichwörter und die dazu gehörende Seitenzahl geschrieben, 
die mit dem Makro \Macro{eintrag}\{Stichwort\}  definiert sind.
\begin{lstlisting}[style=nonumber]
	\newwrite\myfile
	\immediate\openout\myfile=\jobname.sti
	\newcommand{\eintrag}[1]{\immediate\write\myfile{#1:\thepage}}
\end{lstlisting}

Für die spätere gesammelte Anzeige der Stichwörter und die entsprechende Seitenzahl müssen zwei neue Definitionen 
eingeführt werden.
\begin{lstlisting}[style=nonumber]
	\newcommand{\ausgabe}[2]{#1 was referenced on page(s): #2}
	\newcommand{\stichworttabelle}{\IfFileExists{\jobname.tab}{\input{\jobname.tab}}{Tab file not found}}
\end{lstlisting}
Die erste ist dafür zuständig, dass bei dem Macro \Macro{ausgabe}\{Stichwort\} das Stichwort sowie der Text 
\emph{was referenced on page(s):} mit den Seitenzahlen geschrieben werden.
Die zweite Definition gibt die Stichwörter und die Seitenzahlen als Tabelle aus.

Das Ganze noch in die Umgebung \Macro{makeatletter} und \Macro{makeatother} und fertig ist die Vorarbeit in der Prämbel:
\begin{lstlisting}[style=number]
\makeatletter
	\newwrite\myfile
	\immediate\openout\myfile=\jobname.sti
	\newcommand{\eintrag}[1]{\immediate\write\myfile{#1:\thepage}}
	\newcommand{\ausgabe}[2]{S.\ #2}
	\newcommand{\stichworttabelle}{\IfFileExists{\jobname.tab}{\input{\jobname.tab}}{Tab file not found}}
\makeatother
\end{lstlisting}
%%%	\newcommand{\ausgabe}[2]{#1 was referenced on page(s): #2}
Der Python-Code setzt sich folgendermaßen zusammen:


\section{Integration}
Im \TeX-Dokument muss nun an allen Stellen, 
an denen das Stichwort auftaucht oder der Rückverweis stattfinden soll,
der Befehl  \Macro{eintrag}\{Stichwort\} gesetzt werden.
Wenn alle Stichwörter gesetzt sind, 
kann man die Art der Ausgabe wählen.
Entweder werden alle Seitenzahlen zu einem Stichwort angezeigt, dies erfolgt über \Macro{ausgabe}\{Stichwort\}\{\}
oder Stichwörter und Seiten werden mit  \Macro{stichworttabelle} tabellerisch aufgelistet.


Für unser konkretes Beispiel der Katalogeinträge bedeutet dies, dass das Makro \Macro{ausgabe} in die Katalogeintragdefinition integriert werden muss.
\begin{lstlisting}[style=number]
\define@key{catalogue}{literature}{\def\KVliterature{\ausgabe{#1}}}
\end{lstlisting}
Damit wird der Wert, der im \Macro{catalogueentry} unter \Command{literature} angegeben ist (hier \emph{Haus}), an den Befehl \Macro{ausgabe} übergeben:
\begin{lstlisting}[style=number]
\catalogueentry{%
	literature={Haus},
}
\end{lstlisting}
Das bedeutet selbstverständlich, dass im Fließtext entsprechend \Macro{eintrag}\{Haus\} gesetzt werden muss.

Wer es etwas einfacher haben möchte, wählt in der Defintion von \Command{literature} einen automatisch einmaligen Wert, 
den man zudem nicht nochmals im Katalog händisch eingeben muss. 
Ein solcher Wert steckt hinter \Command{KVlabel}, welcher im Zuge der \Command{label}-Definition angelegt ist 
(was die Definition eines \Command{labels} erfordert, das sich jedoch bei Katalogen ohnehin anbietet).
\begin{lstlisting}[style=nonumber]
\define@key{catalogue}{literature}{\def\KVliterature{\ausgabe{\KVlabel}}}
\end{lstlisting}
Und für alle, die sich alle Möglichkeiten offen halten wollen:
\begin{lstlisting}[style=number]
\define@key{catalogue}{literature}{%
	\def\KVliterature{%
		\ausgabe{%
				\ifdef{#1}%
					{#1}%
					{\ifdef{\KVlabel}%
							{\KVlabel}
							{}%
					}
				}
		}
}
\end{lstlisting}
Wenn etwas in den Katalogeintrag \Command{literature} steht, dann wird dieser Eintrag übernommen,
ansonsten wird geschaut, ob das \Command{label} definiert ist und wenn dies nicht der Fall ist, bleibt dieser Katalogeintrag undefiniert und es findet auch keine Weiterverarbeitung statt. 

Für die Darstellung des Katalogeintrages muss ebenfalls eine Neudefinition eingeführt werden:
\begin{lstlisting}[style=nonumber]
	\ifdef{\KVliterature}{\item[Erwähnungen] \KVliterature}{}
\end{lstlisting}



Ein vollständiger Katalogeintrag:
\begin{lstlisting}[style=number]
\catalogueentry{%
	house={Haus des M. Fabius Rufus},
	label={haus:M-Fabius-Rufus},
	size={172},
	description={Haus besteht aus mehreren Einzelgebäuden.},
	location={Regio VII, Insula 16, Eingang 17--22.},
	interior={Reicher Fundkomplex.},
	literature={Haus-Fabius},
}
\end{lstlisting}
Nun wird \emph{Haus} an \Macro{ausgabe} übergeben.
Im Folgenden Fall ist dies der Wert vom Katalogeintrag \Command{label}:\eintrag{Haus-des-Wilden-Ebers}
\begin{lstlisting}[style=number]
\catalogueentry{%
	house={Haus des Wilden Ebers},
	label={Haus-des-Wilden-Ebers},
	size={54},
	description={Renovierung nach Erdbeben 62\,n.\,Chr.},
	location={Regio VII, Insula 4, Eingang 48, 43},
}
\end{lstlisting}
Während im ersten Fall im Fließtext an gewünschter Stelle \Macro{eintrag}\{Haus-Fabius\} gesetzt werden muss,
gilt für die zweite Variante \Macro{eintrag}\{Haus-des-Wilden-Ebers\}.\eintrag{Haus-Rufus}
\begin{bsp}
\catalogueentry{%
	house={Haus des M. Fabius Rufus},
	size={172},
	description={Haus besteht aus mehreren Einzelgebäuden.},
	location={Regio VII, Insula 16, Eingang 17--22.},
	interior={Reicher Fundkomplex.},
	literature={Haus-Fabius},
}
\end{bsp}

\begin{bsp}
\catalogueentry{%
	house={Haus des Wilden Ebers},
	label={Haus-des-Wilden-Ebers},
	size={54},
	description={Renovierung nach Erdbeben 62\,n.\,Chr.},
	location={Regio VII, Insula 4, Eingang 48, 43},
}
\end{bsp}
Sobald diese Vorarbeiten getan sind, alle Stichwörter entsprechend mit \Macro{eintrag} gesetzt sind, 
dann muss einmal (mit \XeLaTeX) kompiliert werden,
anschließend der Python-Code im entsprechenden externen Programm ausgeführt und
schließlich  zurück im \TeX-Editor wiederum einmal kompiliert werden.

\stichworttabelle

%%\lstinputlisting[style=number,language={[AlLaTeX]{TeX}}]{\jobname.tex}%LCB:Test
\end{document}